\documentclass[]{elsarticle} %review=doublespace preprint=single 5p=2 column
%%% Begin My package additions %%%%%%%%%%%%%%%%%%%
\usepackage[hyphens]{url}
\usepackage{lineno} % add
\providecommand{\tightlist}{%
  \setlength{\itemsep}{0pt}\setlength{\parskip}{0pt}}

\bibliographystyle{elsarticle-harv}
\biboptions{sort&compress} % For natbib
\usepackage{graphicx}
\usepackage{booktabs} % book-quality tables
%% Redefines the elsarticle footer
%\makeatletter
%\def\ps@pprintTitle{%
% \let\@oddhead\@empty
% \let\@evenhead\@empty
% \def\@oddfoot{\it \hfill\today}%
% \let\@evenfoot\@oddfoot}
%\makeatother

% A modified page layout
\textwidth 6.75in
\oddsidemargin -0.15in
\evensidemargin -0.15in
\textheight 9in
\topmargin -0.5in
%%%%%%%%%%%%%%%% end my additions to header

\usepackage[T1]{fontenc}
\usepackage{lmodern}
\usepackage{amssymb,amsmath}
\usepackage{ifxetex,ifluatex}
\usepackage{fixltx2e} % provides \textsubscript
% use upquote if available, for straight quotes in verbatim environments
\IfFileExists{upquote.sty}{\usepackage{upquote}}{}
\ifnum 0\ifxetex 1\fi\ifluatex 1\fi=0 % if pdftex
  \usepackage[utf8]{inputenc}
\else % if luatex or xelatex
  \usepackage{fontspec}
  \ifxetex
    \usepackage{xltxtra,xunicode}
  \fi
  \defaultfontfeatures{Mapping=tex-text,Scale=MatchLowercase}
  \newcommand{\euro}{€}
\fi
% use microtype if available
\IfFileExists{microtype.sty}{\usepackage{microtype}}{}
\ifxetex
  \usepackage[setpagesize=false, % page size defined by xetex
              unicode=false, % unicode breaks when used with xetex
              xetex]{hyperref}
\else
  \usepackage[unicode=true]{hyperref}
\fi
\hypersetup{breaklinks=true,
            bookmarks=true,
            pdfauthor={},
            pdftitle={Milestone TWO : Exploratory Network Analysis of Clinical Interactions in the ED},
            colorlinks=true,
            urlcolor=blue,
            linkcolor=magenta,
            pdfborder={0 0 0}}
\urlstyle{same}  % don't use monospace font for urls
\setlength{\parindent}{0pt}
\setlength{\parskip}{6pt plus 2pt minus 1pt}
\setlength{\emergencystretch}{3em}  % prevent overfull lines
\setcounter{secnumdepth}{0}
% Pandoc toggle for numbering sections (defaults to be off)
\setcounter{secnumdepth}{0}
% Pandoc header


\usepackage[nomarkers]{endfloat}

\begin{document}
\begin{frontmatter}

  \title{\emph{Milestone TWO} ``:'' Exploratory Network Analysis of Clinical
Interactions in the ED}
    \author[Emory University]{Tommy Flynn\corref{c1}}
   \ead{tjflynn@emory.edu} 
   \cortext[c1]{Corresponding Author}
      \address[Emory University]{Find the GitHub repository for this Milestone at
\url{https://github.com/tommyflynn/Flynn_N741_Project/tree/master/Flynn_Project}}
  
  \begin{abstract}
  Patient acuity in the Emergency Department is triaged at the beginning
  of the care process using the Emergency Severity Index (ESI). Although
  the ESI is widely used and accepted as a validated predictor of ED
  resource consumption, its predictive power has limitations that can
  negatively impact patient flow and safety. An objective measurement of
  individual resource consumption, that passively observes and calulates
  relative patient needs, would allow nurses and administrators to make
  better decisions with staffing and patient assignemnts. This study
  proposes an approach to continuous passive patient acuity monitoring
  using radio-frequency identifaction (RFID) technology and network
  analysis algorithms to define and potentially predict changes in patient
  acuity. The first step in this process is to determine how acuity, as it
  is currently measured, correlates to the network position of patients in
  the clinical interaction network. A strong correlation between network
  position and acuity is expected, and a change in network position is
  presumed to reflect a change in patient condition.
  \end{abstract}
  
 \end{frontmatter}

\section{Research Question \& Specific
Aims}\label{research-question-specific-aims}

Can network analysis of clinical interactions between patients and staff
provide insight into the complex Emergency Department patient care
process?

Aim 1: Create a netowrk graph from the data. (Baby steps)

Aim 2: Create a visually engaging story of ED network development

\section{Background \& Objectives}\label{background-objectives}

In an Emergency Deparmtent (ED), care is delivered over a network of
face to face human interactions. Patients interact with registration
staff, then a triage nurse who may decide to discuss the patient with a
provider, the provider may then interact directly with the patient, and
so on. In this way, the network grows over time, creating a web of care
that may correlate with the amount and quality of care delivered to
individual patients. - \textbf{The purpose of this study is to explore
the network of clinical interactions that take place in the Emergency
Department and describe the raltionship between those network variables
and patient acuity.} To study this relationship, received permission to
analyse existing data that includes the following; the frequency and
duration of all face-to-face interactions (patients, providers, nurses,
technicians, \& administrators) that occured in the ED for 81 12hr
shifts, the location of those interactions, and individual patients'
medical and demographic characteristics including acuity, chief
complaint, gender, age, arrival mode, and disposition. The network
structural characteristics will be assessed in relation to the industry
standard acuity measure, the Emergency Severity Index (ESI), and
potential confounding variables. Using this data will require specific
knowledge of the R statistical packages, network analysis, and data
science. See Tables 1-4 for my learning goals with respective action
items, timeline, and outcomes.

\section{Data}\label{data}

This study applies a secondary data analysis design due to the
exploritory nature of the research aims. Data was made available with
permissions from the originating research team. The purpose of the
original study was to describe contact characteristics between patients
and staff in the ED of a busy urban hospital to inform cross-infection
control measures. Data were collected using a radio-frequency
identification system that triangulated patient and staff (nurses,
providers, and ancillary staff) locations within the ED at Emory
University Hospital Midtwon. The end result of the RFID collection
system is a data set with every participant's location for the duration
they were in the ED. With the location of every individual in the ED, we
can determine when, and where, two individuals came into within 1 meter
of one another while being in the same room. And that is how we define
an interaction. Data for this secondary analysis were collected using a
prospective, longitudinal, observational design with a random sampling
of one day shift and one night shift per week for one year, July 1, 2009
to June 30, 2010. This strategy was chosen to minimize sampling bias
related to seasonal or weekly fluctuations in census, acuity, and ED
staffing changes. Although a total of 104 shifts were observed, the
original research team retained only 81 shifts for reasons related to
issues with the RFID system and study staff sick leave.(Lowery-North et
al. 2013)

\begin{verbatim}
##   [1] "sid"                         "d8"                         
##   [3] "day"                         "year"                       
##   [5] "startd8time"                 "endd8time"                  
##   [7] "shift_num_ampm"              "mon"                        
##   [9] "starttime"                   "endtime"                    
##  [11] "sdatetime"                   "edatetime"                  
##  [13] "duration_observed"           "censusmax_perperson"        
##  [15] "censusmedian_perperson"      "censusmean_perperson"       
##  [17] "shift_num"                   "ID"                         
##  [19] "i"                           "ShiftStart"                 
##  [21] "ShiftEnd"                    "shiftlength"                
##  [23] "Reason_shortShift"           "month"                      
##  [25] "day_of_month"                "ampm"                       
##  [27] "shift_ampm"                  "shift_d8_ampm"              
##  [29] "studyQuarter"                "quarter_year"               
##  [31] "H1N1"                        "weekday"                    
##  [33] "day1"                        "day2"                       
##  [35] "dayofweek"                   "staff"                      
##  [37] "participant_type"            "jobtitle"                   
##  [39] "staffcount"                  "patcount"                   
##  [41] "total_patients"              "census_min"                 
##  [43] "census_max"                  "participationrate"          
##  [45] "patient_patient"             "staff_patient"              
##  [47] "staff_staff"                 "lone_patient"               
##  [49] "lone_staff"                  "total_edges"                
##  [51] "degree"                      "degree_pp"                  
##  [53] "degree_ps"                   "degree_ss"                  
##  [55] "wdegree"                     "WDEGMIN_SS"                 
##  [57] "wdegmin_PP"                  "wdegmin_PS_STAFF"           
##  [59] "wdegmin_PS_pat"              "RELATIVE_DEGREE"            
##  [61] "RELATIVE_DEGREE_PS_PAT"      "RELATIVE_DEGREE_PS_STAFF"   
##  [63] "RELATIVE_DEGREE_SS"          "RELATIVE_DEGREE_PP"         
##  [65] "LOG_WTDEGREE"                "LNLN_WTDEGREE"              
##  [67] "Closeness_Centrality"        "Betweenness_Centrality"     
##  [69] "Clustering_Coefficient"      "Number_of_triangles"        
##  [71] "Eigenvector_Centrality"      "Authority"                  
##  [73] "Hub"                         "Modularity_Class"           
##  [75] "PageRank"                    "Eccentricity"               
##  [77] "Component_ID"                "avg_degree"                 
##  [79] "avg_wdegree"                 "diameter"                   
##  [81] "apl"                         "num_shortest_paths"         
##  [83] "Density"                     "weak_components"            
##  [85] "triangle"                    "Tran"                       
##  [87] "tot_triangles"               "part_hilo"                  
##  [89] "part_high_acuity"            "part_med_age"               
##  [91] "part_female_freq"            "part_black_freq"            
##  [93] "pop_high_acuity"             "pop_med_age"                
##  [95] "pop_female_freq"             "pop_black_freq"             
##  [97] "pop_acuity_hilo"             "pop_med_age_hilo"           
##  [99] "pop_female_freq_hilo"        "pop_black_freq_hilo"        
## [101] "EncounterNum"                "dup_tag"                    
## [103] "dup_samePT"                  "ED_ARRIVAL"                 
## [105] "ed_departure"                "tag_starttime"              
## [107] "d8ARR"                       "d8DEP"                      
## [109] "timeARR"                     "timeDEP"                    
## [111] "tagofftime"                  "CHANGESID"                  
## [113] "tag_problem"                 "durationInED"               
## [115] "duration_UntilTag"           "timeCoveredbyShift"         
## [117] "daysinED"                    "MinutesInED"                
## [119] "hrsinED"                     "arrival_mode"               
## [121] "AGE"                         "Sex"                        
## [123] "Race"                        "Chief_Complaint"            
## [125] "Acuity"                      "Arr_Mode"                   
## [127] "ED_Disposition"              "admitted"                   
## [129] "ILI_Syndrome"                "ILIwMissing"                
## [131] "acuitycat"                   "Botulism_like"              
## [133] "Hemorrahgic_Illness"         "Lymphadenitis"              
## [135] "Localized_Cutaneous_Lesion"  "Gastrointestinal"           
## [137] "Respiratory"                 "Unexplained_death"          
## [139] "Neurological"                "Rash"                       
## [141] "Specific_Infection"          "Fever"                      
## [143] "Musculoskeletal"             "COUNT_OF_MD"                
## [145] "COUNT_OF_RN"                 "COUNT_OF_ST"                
## [147] "Botulism_liken"              "Fevern"                     
## [149] "Gastrointestinaln"           "Hemorrahgic_Illnessn"       
## [151] "Localized_Cutaneous_Lesionn" "Lymphadenitisn"             
## [153] "Musculoskeletaln"            "Neurologicaln"              
## [155] "Rashn"                       "Respiratoryn"               
## [157] "Specific_Infectionn"         "Unexplained_deathn"         
## [159] "N"                           "Syndrome"                   
## [161] "TOTAL"                       "d1"                         
## [163] "d2"                          "d3"                         
## [165] "max"                         "BLACKyn"                    
## [167] "DegreePScat"                 "NObs"                       
## [169] "duration_observed_Mean"      "duration_observed_Median"   
## [171] "StartsDuringPct"             "syndrome1_Other"            
## [173] "syndrome2_GI"                "syndrome3_Respiratory"      
## [175] "syndrome4_Neuro"             "syndrome5_ChestPain"        
## [177] "syndrome6_Musculoskel"       "possiblecombs"              
## [179] "nduration_observed"          "yearhalf"                   
## [181] "CC1"                         "cc2"                        
## [183] "cc3"                         "cc4"                        
## [185] "cc5"                         "cc6"                        
## [187] "cc7"                         "cc8"                        
## [189] "cc9"                         "cc10"                       
## [191] "cc11"                        "cc12"                       
## [193] "musculoskel"                 "pop_respir_pct"             
## [195] "pop_musculoskel_pct"         "pop_chestP_pct"             
## [197] "pop_GI_pct"                  "pop_neuro_pct"              
## [199] "pop_othersyndrome_pct"       "arrive_by_EMS_pct"          
## [201] "LOG_WDEGMIN_PS_PAT"          "ParticipantCat2"            
## [203] "Num_uniqueRoomChanges"
\end{verbatim}

\begin{verbatim}
##  [1] "sid"                  "numshift"             "d8"                  
##  [4] "shift_ampm"           "Reason_shortShift"    "startd8time"         
##  [7] "endd8time"            "shift_d8_ampm"        "quarter"             
## [10] "weekday"              "H1N1"                 "ED_ARRIVAL"          
## [13] "ed_departure"         "durationInED"         "duration_UntilTag"   
## [16] "Sex"                  "Patient_Age_at_Visit" "Race"                
## [19] "Chief_Complaint"      "Acuity"               "Arr_Mode"            
## [22] "ED_Disposition"       "daysinED"             "MinutesInED"         
## [25] "AGE"                  "ILI_Syndrome"         "ILIwMissing"         
## [28] "hrsinED"              "timeCoveredbyShift"   "popnFreqbyshift"     
## [31] "shift_num_ampm"       "Participant_final"
\end{verbatim}

\subsection{Data Wrangling:}\label{data-wrangling}

Now that I have the data, it turns out that the files are too large
(1.5G) for my \textbf{little} computer. Therefore, I will move forward
with a previous subset of the data that was given to me earlier in the
year. Although this is not the full data set, it will suffice while I
learn the process needed to clean, explore, analyze, \& report on the
data with R and Rmarkdown.

\section{Analysis Plan}\label{analysis-plan}

\subsection{Exploratory Analysis}\label{exploratory-analysis}

Descriptive statistics of the network data as well as patient
demographic data will be evaluated for asssumptions of normality. The
data will be skewed in certain predictable ways due to the observed
patient populations. The distribution of study subject demographics will
be described in tabular format, noting irregularities and potential
sources of error.

\subsection{Variables used for final
analysis:}\label{variables-used-for-final-analysis}

\textbf{Network Variables} \textgreater{} - \emph{Patient eigenvector
centrality} (dependant varialbe of interest) \textgreater{} - Network
density \textgreater{} - Network clustering coefficient \textgreater{} -
Network diversity scale??

\textbf{Staff variables} \textgreater{} - Title (RN, MD, staff)

\textbf{Patient variables} \textgreater{} - \emph{Acuity} (ESI,
independant variable of interest) \textgreater{} - Commorbidities
(index) \textgreater{} - Gender \textgreater{} - Age \textgreater{} -
Race \textgreater{} - Ethnicity \textgreater{} - Arrival mode (ambulance
v. walk-in) \textgreater{} - Education (if available) \textgreater{} -
Disposition (admission v. discharge) \textgreater{} - Length of stay
(common measure of quality in the literature used for comparison)
\textgreater{} - Time before first provider contact (common measure of
quality in the literature used for comparison))

\subsection{Analysis}\label{analysis}

We start with discriptive statistics of all relevant variables for
assumptions of normality. Muliple linear regression will be used for the
final analysis to assess the correlation between patient acuity and
patient centrality. Relationships will be evaluated visually (see below)
as well as statistically to an alpha of 0.05.

\subsection{Schedule}\label{schedule}

-Milestone 1: February 14th, 2018 -Milestone 2 UPDATE: March 28th
-Milestone 2: April 4th, 2018 -Final Proposal: April 26th, 2018

\section{Results}\label{results}

Results will be discussed with the visual supplementation of network
graphs. This allows the reader to understand concepts that may be
difficult to grasp through text alone.

\section{Discussion}\label{discussion}

Allocating staff resources in an Emergency Department is an ongoing
challenge. How can these results begin to offer solutions to ED staff
and patient management?

What were my primary limitation (both expected and unexpected)?

\section{Conclusion}\label{conclusion}

Did I meet my learning objectives? How would I design a better study
next time?

\section*{References}\label{references.unnumbered}
\addcontentsline{toc}{section}{References}

\hypertarget{refs}{}
\hypertarget{ref-RN1X}{}
Lowery-North, Douglas W., Vicki Stover Hertzberg, Lisa Elon, George
Cotsonis, Sarah A. Hilton, II Vaughns Christopher F., Eric Hill, Alok
Shrestha, Alexandria Jo, and Nathan Adams. 2013. ``Measuring Social
Contacts in the Emergency Department.'' Journal Article. \emph{PLoS ONE}
8 (8): e70854.
doi:\href{https://doi.org/10.1371/journal.pone.0070854}{10.1371/journal.pone.0070854}.

\end{document}


