\documentclass[]{elsarticle} %review=doublespace preprint=single 5p=2 column
%%% Begin My package additions %%%%%%%%%%%%%%%%%%%
\usepackage[hyphens]{url}
\usepackage{lineno} % add
\providecommand{\tightlist}{%
  \setlength{\itemsep}{0pt}\setlength{\parskip}{0pt}}

\bibliographystyle{elsarticle-harv}
\biboptions{sort&compress} % For natbib
\usepackage{graphicx}
\usepackage{booktabs} % book-quality tables
%% Redefines the elsarticle footer
%\makeatletter
%\def\ps@pprintTitle{%
% \let\@oddhead\@empty
% \let\@evenhead\@empty
% \def\@oddfoot{\it \hfill\today}%
% \let\@evenfoot\@oddfoot}
%\makeatother

% A modified page layout
\textwidth 6.75in
\oddsidemargin -0.15in
\evensidemargin -0.15in
\textheight 9in
\topmargin -0.5in
%%%%%%%%%%%%%%%% end my additions to header

\usepackage[T1]{fontenc}
\usepackage{lmodern}
\usepackage{amssymb,amsmath}
\usepackage{ifxetex,ifluatex}
\usepackage{fixltx2e} % provides \textsubscript
% use upquote if available, for straight quotes in verbatim environments
\IfFileExists{upquote.sty}{\usepackage{upquote}}{}
\ifnum 0\ifxetex 1\fi\ifluatex 1\fi=0 % if pdftex
  \usepackage[utf8]{inputenc}
\else % if luatex or xelatex
  \usepackage{fontspec}
  \ifxetex
    \usepackage{xltxtra,xunicode}
  \fi
  \defaultfontfeatures{Mapping=tex-text,Scale=MatchLowercase}
  \newcommand{\euro}{€}
\fi
% use microtype if available
\IfFileExists{microtype.sty}{\usepackage{microtype}}{}
\ifxetex
  \usepackage[setpagesize=false, % page size defined by xetex
              unicode=false, % unicode breaks when used with xetex
              xetex]{hyperref}
\else
  \usepackage[unicode=true]{hyperref}
\fi
\hypersetup{breaklinks=true,
            bookmarks=true,
            pdfauthor={},
            pdftitle={Milestone 1 Testing a Model for an Automated Real-Time Acuity Monitoring System in the Emergency Department},
            colorlinks=true,
            urlcolor=blue,
            linkcolor=magenta,
            pdfborder={0 0 0}}
\urlstyle{same}  % don't use monospace font for urls
\setlength{\parindent}{0pt}
\setlength{\parskip}{6pt plus 2pt minus 1pt}
\setlength{\emergencystretch}{3em}  % prevent overfull lines
\setcounter{secnumdepth}{0}
% Pandoc toggle for numbering sections (defaults to be off)
\setcounter{secnumdepth}{0}
% Pandoc header


\usepackage[nomarkers]{endfloat}

\begin{document}
\begin{frontmatter}

  \title{Milestone 1 Testing a Model for an Automated Real-Time Acuity Monitoring
System in the Emergency Department}
    \author[Emory University]{Tommy Flynn\corref{c1}}
   \ead{tjflynn@emory.edu} 
   \cortext[c1]{Corresponding Author}
      \address[Emory University]{Emory University Nell Hodgson School of Nursing, 1520 Clifton Road NE,
Atlanta, GA, 30322}
  
  \begin{abstract}
  The purpose of this project is to determine if patient acuity in the ED
  is correlated to patient eigenvector centrality in the network of all
  face to face ED interactions.
  \end{abstract}
  
 \end{frontmatter}

\section{Overview \& Motivation}\label{overview-motivation}

\begin{itemize}
\tightlist
\item
  Overview and Motivation: Why did you undertake this particular
  project? What inspired you, what are your background and research
  interests that may have influenced your decision?
\item
  Project Objectives: What is the primary focal question that you are
  trying to answer? What would you like to learn and accomplish?
\item
  Data: from where and how are you acquiring your data? \textbf{Provide
  a link to your data source.}
\item
  Data Wrangling: Do you anticipate that there will be extensive data
  cleaning / reshaping / extraction? Are there questions you will need
  to calculate in your data (e.g., perhaps you have height and weight,
  but not BMI)? How will you implement this particular data wrangling
  step?
\item
  Exploratory Analysis: Which methods / visualizations are you planning
  to use to explore your tidy dataset?
\item
  Analysis: How are you planning to analyze your data? \_ Schedule,
  keeping in mind the due dates listed above for the intermediate and
  final milestones, make a plan to meet these deadlines. Write these in
  terms of weekly tasks / goals.
\end{itemize}

As a ballpark, your proposal should be about 2-3 pages of text, along
with shells of the tables and figures that you plan. You could even
include some preliminary data acquisition / analysis steps.

After we receive your proposals we will find a time to meet with you to
discuss your proposal and also to help guide you through the rest of the
analysis.

\paragraph{Objectives}\label{objectives}

The purpose of this study is to test the accuracy of the Emergency
Severity Index (ESI) as a metric used to predict patient needs. To test
the accuracy of the ESI, this study uses radiofrequency identification
(RFID) locating data that includes every interaction patients have with
ED staff (nurses, techs, administrators, and providers) and the duration
of each of those interactions. Using this data will require specific
knowledge of the R statistical packages, network analysis, and data
science. My learning objectives are to: - Use Github, R Studio, R
Markdown, to clean, describe, analyze, and publish results from the
identified database. - Apply visualization tools to the data and results
that create greater understanding of the research and significance. -

\paragraph{Functionality}\label{functionality}

The Elsevier article class is based on the standard article class and
supports almost all of the functionality of that class. In addition, it
features commands and options to format the

\begin{itemize}
\item
  document style
\item
  baselineskip
\item
  front matter
\item
  keywords and MSC codes
\item
  theorems, definitions and proofs
\item
  lables of enumerations
\item
  citation style and labeling.
\end{itemize}

\section{Methods}\label{methods}

The author names and affiliations could be formatted in two ways:

\begin{enumerate}
\def\labelenumi{(\arabic{enumi})}
\item
  Group the authors per affiliation.
\item
  Use footnotes to indicate the affiliations.
\end{enumerate}

See the front matter of this document for examples. You are recommended
to conform your choice to the journal you are submitting to.

\section{Results}\label{results}

\section{Discussion}\label{discussion}

\section{Conclusion}\label{conclusion}

There are various bibliography styles available. You can select the
style of your choice in the preamble of this document. These styles are
Elsevier styles based on standard styles like Harvard and Vancouver.
Please use BibTeX~to generate your bibliography and include DOIs
whenever available.

Here are two sample references: Allaire et al. (2017; R Core Team 2017)

\section*{References}\label{references.unnumbered}
\addcontentsline{toc}{section}{References}

\hypertarget{refs}{}
\hypertarget{ref-Allaire2017}{}
Allaire, JJ, R Foundation, Hadley Wickham, Journal of Statistical
Software, Yihui Xie, Ramnath Vaidyanathan, Association for Computing
Machinery, et al. 2017. \emph{Rticles: Article Formats for R Markdown}.
\url{https://CRAN.R-project.org/package=rticles}.

\hypertarget{ref-CRAN}{}
R Core Team. 2017. \emph{R: A Language and Environment for Statistical
Computing}. Vienna, Austria: R Foundation for Statistical Computing.
\url{https://www.R-project.org/}.

\end{document}


