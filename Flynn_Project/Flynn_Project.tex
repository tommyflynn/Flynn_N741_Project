\documentclass[]{elsarticle} %review=doublespace preprint=single 5p=2 column
%%% Begin My package additions %%%%%%%%%%%%%%%%%%%
\usepackage[hyphens]{url}
\usepackage{lineno} % add
\providecommand{\tightlist}{%
  \setlength{\itemsep}{0pt}\setlength{\parskip}{0pt}}

\bibliographystyle{elsarticle-harv}
\biboptions{sort&compress} % For natbib
\usepackage{graphicx}
\usepackage{booktabs} % book-quality tables
%% Redefines the elsarticle footer
%\makeatletter
%\def\ps@pprintTitle{%
% \let\@oddhead\@empty
% \let\@evenhead\@empty
% \def\@oddfoot{\it \hfill\today}%
% \let\@evenfoot\@oddfoot}
%\makeatother

% A modified page layout
\textwidth 6.75in
\oddsidemargin -0.15in
\evensidemargin -0.15in
\textheight 9in
\topmargin -0.5in
%%%%%%%%%%%%%%%% end my additions to header

\usepackage[T1]{fontenc}
\usepackage{lmodern}
\usepackage{amssymb,amsmath}
\usepackage{ifxetex,ifluatex}
\usepackage{fixltx2e} % provides \textsubscript
% use upquote if available, for straight quotes in verbatim environments
\IfFileExists{upquote.sty}{\usepackage{upquote}}{}
\ifnum 0\ifxetex 1\fi\ifluatex 1\fi=0 % if pdftex
  \usepackage[utf8]{inputenc}
\else % if luatex or xelatex
  \usepackage{fontspec}
  \ifxetex
    \usepackage{xltxtra,xunicode}
  \fi
  \defaultfontfeatures{Mapping=tex-text,Scale=MatchLowercase}
  \newcommand{\euro}{€}
\fi
% use microtype if available
\IfFileExists{microtype.sty}{\usepackage{microtype}}{}
\ifxetex
  \usepackage[setpagesize=false, % page size defined by xetex
              unicode=false, % unicode breaks when used with xetex
              xetex]{hyperref}
\else
  \usepackage[unicode=true]{hyperref}
\fi
\hypersetup{breaklinks=true,
            bookmarks=true,
            pdfauthor={},
            pdftitle={Milestone TWO Exploratory Network Analysis of Clinical Interactions in the ED},
            colorlinks=true,
            urlcolor=blue,
            linkcolor=magenta,
            pdfborder={0 0 0}}
\urlstyle{same}  % don't use monospace font for urls
\setlength{\parindent}{0pt}
\setlength{\parskip}{6pt plus 2pt minus 1pt}
\setlength{\emergencystretch}{3em}  % prevent overfull lines
\setcounter{secnumdepth}{0}
% Pandoc toggle for numbering sections (defaults to be off)
\setcounter{secnumdepth}{0}
% Pandoc header


\usepackage[nomarkers]{endfloat}

\begin{document}
\begin{frontmatter}

  \title{\emph{Milestone TWO} Exploratory Network Analysis of Clinical
Interactions in the ED}
    \author[Emory University]{Tommy Flynn\corref{c1}}
   \ead{tjflynn@emory.edu} 
   \cortext[c1]{Corresponding Author}
      \address[Emory University]{GitHub repository
\url{https://github.com/tommyflynn/Flynn_N741_Project/tree/master/Flynn_Project}}
  
  \begin{abstract}
  Patient acuity in the Emergency Department is triaged at the beginning
  of the care process using the Emergency Severity Index (ESI). Although
  the ESI is widely used and accepted as a validated predictor of ED
  resource consumption, its predictive power has limitations that can
  negatively impact patient flow and safety. An objective measurement of
  individual resource consumption, that passively observes and calulates
  relative patient need in real time would allow charge nurses and
  administrators to make informed decisions for more effective and
  efficient patient care. This study tests a novel approach to patient
  acuity monitoring using real-time location system (RTLS) data and
  network analysis. The first step in this process is to determine how
  acuity, as it is currently measured, correlates to network structural
  development over time in the clinical interaction network.
  \end{abstract}
  
 \end{frontmatter}

\section{Research Question \& Specific
Aims}\label{research-question-specific-aims}

Can network analysis of clinical interactions between patients and staff
provide insight into the complex Emergency Department patient care
process? (Canto et al. 2000)

Aim 1: Explore the network of clinical interactions in the ED between
patients and staff to determine whether predictable patterns emerge in
terms of centrality, density, and change over time.

Aim 2: Test the assocaition between patient acuity and network position
measure of eigenvector centrality of patient composite network, compared
to the centrality of teh dynamic patient network (measure TBD).

\section{Background \& Objectives}\label{background-objectives}

Emergency Severity Index (ESI) is a validated metric used to triage
patients in the . to .(Tanabe et al. 2004) That triage nurse may decide
to involve the charge nurse or a physician given various concerns about
the patient. These interactions, observed and measured by the Real Time
Location System (RTLS), continue as more patients are triaged, moved
into patient rooms, and so on toward a vast and complex network of
interactions. This web of care is likely to correlate with the amount
and quality of care delivered to individual patients. - \textbf{The
purpose of this study is to explore the network of clinical interactions
that take place in the Emergency Department and describe the raltionship
between those network variables and patient acuity.} To study this
relationship, received permission to analyse existing data that includes
the following; the frequency and duration of all face-to-face
interactions (patients, providers, nurses, technicians, \&
administrators) that occured in the ED for 81 12hr shifts, the location
of those interactions, and individual patients' medical and demographic
characteristics including acuity, chief complaint, gender, age, arrival
mode, and disposition. The network structural characteristics will be
assessed in relation to the industry standard acuity measure, the
Emergency Severity Index (ESI), and potential confounding variables.
Using this data will require specific knowledge of the R statistical
packages, network analysis, and data science. See Tables 1-4 for my
learning goals with respective action items, timeline, and outcomes.

\section{Data}\label{data}

This study applies a secondary data analysis design due to the
exploritory nature of the research aims. Data was made available with
permissions from the originating research team. The purpose of the
original study was to describe contact characteristics between patients
and staff in the ED of a busy urban hospital to inform cross-infection
control measures. Data were collected using a radio-frequency
identification system that triangulated patient and staff (nurses,
providers, and ancillary staff) locations within the ED at Emory
University Hospital Midtwon. Data for this secondary analysis were
collected using a prospective, longitudinal, observational design with a
random sampling of one day shift and one night shift per week for one
year, July 1, 2009 to June 30, 2010. This strategy was chosen to
minimize sampling bias related to seasonal or weekly fluctuations in
census, acuity, and ED staffing changes. Although a total of 104 shifts
were observed, the original research team retained only 81 shifts for
reasons related to issues with the RFID system and study staff sick
leave.(Lowery-North et al. 2013)

\subsection{Data Wrangling:}\label{data-wrangling}

I have requested the original/raw data, which will require cleaning and
organizing to meet the needs of my research aims. Data will be
maintained in private repositories in the GitHub version control
platform. Patient characteristic data will be evaluated for missing or
implausible data with discriptive analyses, and RFID generated networks
will be included for statistical analysis if variables of network
density, centrality, and a network diversity scale are distributed
normally across networks.

\section{Analysis Plan}\label{analysis-plan}

\subsection{Exploratory Analysis}\label{exploratory-analysis}

Descriptive statistics of the network data as well as patient
demographic data will be evaluated for asssumptions of normality. The
data will be skewed in certain predictable ways due to the observed
patient populations. The distribution of study subject demographics will
be described in tabular format, noting irregularities and potential
sources of error.

\subsection{Variables available for final
analysis:}\label{variables-available-for-final-analysis}

\textbf{Network Variables} \textgreater{} - Network Centrality (based on
the eigenvector up to, but not including, any other patient-staff
interactions) \textgreater{} - Network density \textgreater{} - Network
clustering coefficient

\textbf{Staff title} \textgreater{} - Title (RN, MD, Other Staff)

\textbf{Patient variables} \textgreater{} - \emph{Acuity} (ESI,
independant variable of interest) \textgreater{} - Gender \textgreater{}
- Age \textgreater{} - Race \textgreater{} - Arrival mode (ambulance v.
walk-in) \textgreater{} - Disposition (admission v. discharge)
\textgreater{} - Length of stay (common measure of quality in the
literature used for comparison)

\subsection{Analysis}\label{analysis}

The open-source R statistical language and R-Studio user interface from
the developers at CRAN were used for all data exploration, wrangling,
cleaning, description, and analysis.(R Core Team 2017) Pandoc's Markdown
allows for seemless integration of code, results, visualizations, and
author interpretation of the research into a single document.(Allaire et
al. 2017) Running all code and calculating all results within the
menuscript itself, Markdown eliminates risk for errors in transferring
statistical software output into foreign documents. The data were
explored, cleaned, and assessed for statistical assumptions using the
Tidyverse group of R packages.(Wickham 2017, Wickham (2016)) Data were
prepared for network analysis with the iGraph package.(Csardi and Nepusz
2006) Muliple linear regression will be used for the final analysis to
assess the correlation between patient acuity and patient centrality.
Relationships will be evaluated visually (see below) as well as
statistically to an alpha of 0.05.

\section{Results}\label{results}

Results will be discussed with the visual supplementation of network
graphs. This allows the reader to understand concepts that may be
difficult to grasp through text alone.

\section{Discussion}\label{discussion}

Allocating staff resources in an Emergency Department is an ongoing
challenge. How can these results begin to offer solutions to ED staff
and patient management?

What were my primary limitation (both expected and unexpected)?

\section{Conclusion}\label{conclusion}

Did I meet my learning objectives? How would I design a better study
next time?

\section*{References}\label{references.unnumbered}
\addcontentsline{toc}{section}{References}

\hypertarget{refs}{}
\hypertarget{ref-MARKDOWN}{}
Allaire, JJ, Jeffrey Horner, Vicent Marti, and Natacha Porte. 2017.
\emph{Markdown: 'Markdown' Rendering for R}.
\url{https://CRAN.R-project.org/package=markdown}.

\hypertarget{ref-RN602}{}
Canto, John G., Jeroan J. Allison, Catarina I. Kiefe, Contessa Fincher,
Robert Farmer, Padmini Sekar, Sharina Person, and Norman W. Weissman.
2000. ``Relation of Race and Sex to the Use of Reperfusion Therapy in
Medicare Beneficiaries with Acute Myocardial Infarction.'' Journal
Article. \emph{New England Journal of Medicine} 342 (15): 1094--1100.
doi:\href{https://doi.org/10.1056/NEJM200004133421505}{10.1056/NEJM200004133421505}.

\hypertarget{ref-IGRAPH}{}
Csardi, Gabor, and Tamas Nepusz. 2006. ``The Igraph Software Package for
Complex Network Research.'' \emph{InterJournal} Complex Systems: 1695.
\url{http://igraph.org}.

\hypertarget{ref-RN1X}{}
Lowery-North, Douglas W., Vicki Stover Hertzberg, Lisa Elon, George
Cotsonis, Sarah A. Hilton, II Vaughns Christopher F., Eric Hill, Alok
Shrestha, Alexandria Jo, and Nathan Adams. 2013. ``Measuring Social
Contacts in the Emergency Department.'' Journal Article. \emph{PLoS ONE}
8 (8): e70854.
doi:\href{https://doi.org/10.1371/journal.pone.0070854}{10.1371/journal.pone.0070854}.

\hypertarget{ref-CRAN}{}
R Core Team. 2017. \emph{R: A Language and Environment for Statistical
Computing}. Vienna, Austria: R Foundation for Statistical Computing.
\url{https://www.R-project.org/}.

\hypertarget{ref-RN251}{}
Tanabe, Paula, Rick Gimbel, Paul R. Yarnold, and James G. Adams. 2004.
``The Emergency Severity Index (Version 3) 5-Level Triage System Scores
Predict Ed Resource Consumption.'' Journal Article. \emph{Journal of
Emergency Nursing} 30 (1): 22--29.
doi:\href{https://doi.org/http://dx.doi.org/10.1016/j.jen.2003.11.004}{http://dx.doi.org/10.1016/j.jen.2003.11.004}.

\hypertarget{ref-GGPLOT2}{}
Wickham, Hadley. 2016. \emph{Ggplot2: Elegant Graphics for Data
Analysis}. Springer-Verlag New York. \url{http://ggplot2.org}.

\hypertarget{ref-TIDY}{}
---------. 2017. \emph{Tidyverse: Easily Install and Load the
'Tidyverse'}. \url{https://CRAN.R-project.org/package=tidyverse}.

\end{document}


